\chapter{Results}
\section{SUS Results}
The application received an average SUS-score of 82.5 which is considered to be a Good score according to the numerical SUS Scale. Below tables represent a breakdown of the SUS calculations. 
\subsection{Questions and Average rating}

\begin{tabularx}{\textwidth}{ || >{\raggedright\arraybackslash}X2 || >{\centering\arraybackslash}X || }
\hline
\hline
\textbf{Question} & \textbf{Average rating} \\
\hline
\hline
I think that I would like to use this system frequently & 4.2 \\
\hline
I found the system unnecessarily complex & 1 \\
\hline
I thought the system was easy to use & 4.2 \\
\hline
I think that I would need the support of a technical person to be able to use this system & 1.6 \\
\hline
I found the various functions in this system were well integrated & 4.2 \\
\hline
I thought there was too much inconsistency in this system & 1.4 \\
\hline
I would imagine that most people would learn to use this system very quickly & 4 \\
\hline
I found the system very awkward to use & 1.6 \\
\hline
I felt very confident using the system & 3.6 \\
\hline
I needed to learn a lot of things before I could get going with the system & 1.6 \\
\hline
\hline
\end{tabularx}

\subsection{Participants SUS-score}
\begin{tabularx}{\textwidth}{ || >{\raggedright\arraybackslash}X2 || >{\centering\arraybackslash}X || }
\hline
\hline
\textbf{Participant} & \textbf{SUS score} \\
\hline
\hline
1 & 90.0 \\
\hline
2 & 92.5 \\
\hline
3 & 60.0 \\
\hline
4 & 80.0 \\
\hline
5 & 90.0 \\
\hline
\hline
\textbf{Average SUS score} & 82.5 \\
\hline
\textbf{Standard deviation} & 13.5 \\
\hline
\hline
\end{tabularx}
\section{Key comments}
\begin{tabularx}{1\textwidth}{ || >{\centering\arraybackslash}X | >{\centering\arraybackslash}X || }
\hline
\hline
\textbf{POSITIVE} & \textbf{CONSTRUCTIVE} \\
\hline
\hline
I strongly like the fact that the application has a minimalistic design & I think that the icons for Goals and History might be improved \\
\hline
The icons for the Home and Profile page are optimal and represent the content of the pages good & I think it's vital to have a initial guide in the beginning to explain what Streaks are and how to navigate \\
\hline
I believe that the navigation felt natural and I liked the fact that there was not too many options to click on after entering a page & I thought that achievements could be found under the History page \\
\hline
\hline
\end{tabularx}

\section{Tests}
\subsection{Test 1}

\noindent\rule{15.1cm}{0.4pt}\\
\\
\begin{tabularx}{0.6\textwidth}{ >{\raggedright\arraybackslash}X  >{\raggedright\arraybackslash}X  }
\textbf{Date}: & 21/11  \\
\textbf{Location}: & Remote  \\
\textbf{Test Leader}: & Matthew Soulaka  \\
\textbf{Test Person} & Participant 1, Female 33  \\
\textbf{SUS Score} & 90/100  \\

\end{tabularx}\\
\\
\textbf{Summarization from scenarios and questions} \\
\noindent Overall the participant had no major issues with navigating through the application and completing the given tasks. The participant commented that the overall logic and navigation felt natural and nothing was out of the ordinary. The participant also liked the trees and gamification features of the application. This is also reflected in the high SUS-score of 90. However, the participant believes that the application needs more color and it should come with an initial guide so that the users can get familiar with the nav-bar icons and what they represent.\\

\noindent\textbf{Results from test scenarios}
\begin{enumerate}
\item The participant understood quickly that the password can be changed under the profile-page
\subitem \textbf{Amount of clicks}: average
\item The participant quickly navigated to the Home page and pressed on add new val
\subitem \textbf{Amount of clicks:} average
\item The participant immediately saw the pen icon representing the edit-function and clicked
on it
\subitem \textbf{Amount of clicks}: average
\item The participant clicked on the profile page and after reading the possible alternatives pressed on the relatives-button
\subitem \textbf{Amount of clicks}: average
\item The participant first clicked on the home page and tried to click around there to find goals but realized after a while that they have to try to click on the star icon 
\subitem \textbf{Amount of clicks}: more than average
\item The participant first clicked on the home page and tried to find alternatives but realized after a while that the second icon on the nav-bar is the way to go
\subitem \textbf{Amount of clicks}: more than average
\item The participant remembered that goals could be found under the star icon and found achievements there too
\subitem \textbf{Amount of clicks:} average 
\item The participant immediately clicked on the home page and then on the tree 
\subitem \textbf{Amount of clicks}: average 
\item The participant quickly navigated to the correct button
\subitem \textbf{Amount of clicks}: average 
\item The participant quickly navigated to the correct page
\subitem \textbf{Amount of clicks}: average 
\end{enumerate}
	




\subsection{Test 2}

\noindent\rule{15.1cm}{0.4pt}\\
\\
\begin{tabularx}{0.6\textwidth}{ >{\raggedright\arraybackslash}X  >{\raggedright\arraybackslash}X  }
\textbf{Date}: & 21/11  \\
\textbf{Location}: & Remote  \\
\textbf{Test Leader}: & Matthew Soulaka  \\
\textbf{Test Person} & Participant 2, Female 38  \\
\textbf{SUS Score} & 92.5/100  \\

\end{tabularx}\\
\\
\textbf{Summarization from scenarios and questions} \\
\noindent Overall the participant had no major issues with navigating through the application and completing the given tasks. The participant commented that the overall logic and navigation felt natural and nothing was out of the ordinary. The participant strongly liked the minimalist design features of the application.All of this is reflected in the high SUS-score of 92.5. However, the participant believes that the application needs more gamification and a better icon for the history page (or a guide that teaches users what each icon stands for). The participant had a relatively good understanding of what streaks stands for and compared it with Duolingo.\\

\noindent\textbf{Results from test scenarios}
\begin{enumerate}
\item This task was completed with no issues
\subitem \textbf{Amount of clicks}: average
\item This task was completed with no issues
\subitem \textbf{Amount of clicks}: average
\item The test participant struggled with finding the edit-function and clicked around in the home-page and profile-page for a while before realizing where the pen icon was. The participant would have wanted more indications of where one could change previously imputed values. 
\subitem \textbf{Amount of clicks}: more than average
\item This task was completed with no issues
\subitem \textbf{Amount of clicks}: average
\item The participant clicked first on the home-page but then navigated to the Goals-page after realizing that no goals could be found on the goal page
\subitem \textbf{Amount of clicks}: more than average
\item The participant clicked on the home-page and then the profile-page before realizing where to actually find the previously entered measurements. The participant commented that the icon for the history page might be improved
\subitem \textbf{Amount of clicks}: more than average
\item The participant clicked on the goals page and found the goal with no issues
\subitem \textbf{Amount of clicks}: average
\item This task was completed with no issues
\subitem \textbf{Amount of clicks}: average
\item This task was completed with no issues
\subitem \textbf{Amount of clicks}: average
\item This task was completed with no issues 
\subitem \textbf{Amount of clicks}: average
\end{enumerate}



\subsection{Test 3}

\noindent\rule{15.1cm}{0.4pt}\\
\\
\begin{tabularx}{0.6\textwidth}{ >{\raggedright\arraybackslash}X  >{\raggedright\arraybackslash}X  }
\textbf{Date}: & 21/11  \\
\textbf{Location}: & Remote  \\
\textbf{Test Leader}: & Matthew Soulaka  \\
\textbf{Test Person} & Participant 3, Male 56  \\
\textbf{SUS Score} & 90/100  \\

\end{tabularx}\\
\\
\textbf{Summarization from scenarios and questions} \\
\noindent Overall the participant had no major issues with navigating through the application and completing the given tasks. The participant had trouble with some of the tasks and did not fully understand them but could with some explanations proceed as normal. The participant commented that the overall logic and navigation felt natural and nothing was out of the ordinary. The participant strongly liked the minimalist design features of the application and believes that they are a major key to a successful application in these settings. All of this is reflected in the high SUS-score of 90. However, the participant believes that the icon for the goals page could be misinterpreted and does not reflect what it’s content is and should therefore be seen over (or implement  a guide that teaches users what each icon stands for).  The participant had an good understanding of what streaks mean.\\

\noindent\textbf{Results from test scenarios}
\begin{enumerate}
\item This task was completed with no issues
\subitem \textbf{Amount of clicks}: average
\item This task was completed with no issues
\subitem \textbf{Amount of clicks}: average
\item This task was completed with no issues
\subitem \textbf{Amount of clicks}: average
\item This task was completed with no issues
\subitem \textbf{Amount of clicks}: average
\item The participant clicked first on the home-page and clicked around with no results. The participant then managed to navigate correctly after some guidance from the test leader. The participant commented that the star-icon was not something he believed is an optimal icon and should be improved. 
\subitem \textbf{Amount of clicks}: more than average
\item The participant clicked on the home-page before realizing where to actually find the previously entered measurements. The participant commented that the icon for the history page can be improved.
\subitem \textbf{Amount of clicks}: more than average
\item The participant navigated first to the home-page and realized that nothing could be found there and after a while clicked on the goals page where achievements were found immediately. 
\subitem\textbf{Amount of clicks}: more than average
\item This task was completed with no issues 
\subitem \textbf{Amount of clicks}: average
\item The participant had a hard time understanding what gamification was in this context and needed some guidance before quickly navigating to the correct page. 
\subitem \textbf{Amount of clicks}: more than average
\item This task was completed with no issues 
\subitem \textbf{Amount of clicks}: average
\end{enumerate}



\subsection{Test 4}

\noindent\rule{15.1cm}{0.4pt}\\
\\
\begin{tabularx}{0.6\textwidth}{ >{\raggedright\arraybackslash}X  >{\raggedright\arraybackslash}X  }
\textbf{Date}: & 21/11  \\
\textbf{Location}: & Remote  \\
\textbf{Test Leader}: & Fredrik Thorsson  \\
\textbf{Test Person} & Participant 4, Male 56  \\
\textbf{SUS Score} & 60/100  \\

\end{tabularx}\\
\\
\textbf{Summarization from scenarios and questions} \\
\noindent It was generally hard for the test person to overview the application and its content. Icons didn’t match the participants' recognition, which made it hard to navigate through the application. The language was a contributory factor for the difficulties the participant encountered. Words as gamification and streaks were completely new to the test person. The star icon was associated with favorites.\\

\noindent\textbf{Results from test scenarios}
\begin{enumerate}
\item The participant understood the icon for the profile page and could easily change his password.
\subitem \textbf{Amount of clicks}: average
\item The participant could instantly see the upcoming measurement on the home page to enter a new value for blood sugar levels.
\subitem \textbf{Amount of clicks}: average
\item The participant finds it easy to edit his done measurement from the home page.
\subitem \textbf{Amount of clicks}: average
\item The participant had because of lacking english skills problems with finding the relative settings. After explanation the participant thought it was logic. 
\subitem \textbf{Amount of clicks}: more than average
\item The participant didn’t associate the star icon with goals. Instead he thought it was somehow related to favorites. 
\subitem \textbf{Amount of clicks}: more than average
\item The participant easily found the history page and could with no problems find this registered activity value. 
\subitem \textbf{Amount of clicks}: average
\item Again the participant had problems with associating the star with goals. After entering the goals page the participant had some issues with locating the achievements button.
\subitem \textbf{Amount of clicks}: more than average
\item The participant went to the help page to find information about the help page. He searched for information explaining what the trees are used for. He didn’t think it was sufficient information just clicking on the oak sapling. 
\subitem \textbf{Amount of clicks}: more than average
\item The participant could easily find where to change the gamification levels.
\subitem \textbf{Amount of clicks}: average
\item Again the participant had issues with the star icon. After entering the goals page the participant first went to achievements. 
\subitem \textbf{Amount of clicks}: more than average
\end{enumerate}


\subsection{Test 5}

\noindent\rule{15.1cm}{0.4pt}\\
\\
\begin{tabularx}{0.6\textwidth}{ >{\raggedright\arraybackslash}X  >{\raggedright\arraybackslash}X  }
\textbf{Date}: & 21/11  \\
\textbf{Location}: & Remote  \\
\textbf{Test Leader}: & Fredrik Thorsson  \\
\textbf{Test Person} & Participant 4, Male 56  \\
\textbf{SUS Score} & 80/100  \\

\end{tabularx}\\
\\
\textbf{Summarization from scenarios and questions} \\
\noindent Would like to use the application if there was a need, like the purpose. Could take some time to learn the application, but after using it a while the layout feels logical. Understood the term streak. Had never heard the word gamification before. The icons for history and goals don't feel logical, relating the star to favorites. The layout of the home page together with the navigation bar gives the user a good overview.  \\

\noindent\textbf{Results from test scenarios}
\begin{enumerate}
\item The participant had no problems with changing the password entering the profile page instantly. 
\subitem \textbf{Amount of clicks}: average
\item The participant easily found the upcoming measurements.
\subitem \textbf{Amount of clicks}: average
\item The participant easily found the edit icon for done measurements. 
\subitem \textbf{Amount of clicks}: average
\item The participant easily navigated to the profile page and found the settings for authorized relatives.
\subitem \textbf{Amount of clicks}: average
\item The participant didn’t associate the star icon with goals. However, after finding the goals page he could easily find the related goals to diabetes. 
\subitem \textbf{Amount of clicks}: average
\item The participant didn’t agree with the chosen icon for history, however he could find the page and see his registered activity values.
\subitem \textbf{Amount of clicks}: average
\item The participant first went to the history page and the activity subpage. After entering the goals page the participant could easily find the achievement page.
\subitem \textbf{Amount of clicks}: more than average
\item There was some confusion if the information about the trees were located in the help section or after clicking the tree. However, both locations could easily be found.
\subitem \textbf{Amount of clicks}: average
\item Found the settings for gamification on the profile page without problems.
\subitem \textbf{Amount of clicks}: average
\item Again, the participant went to the history page. After navigating to the goals page he clicked on achievements. After going back he found the goals for Lose 5 kg. 
\subitem \textbf{Amount of clicks}: more than average
\end{enumerate}



